 \documentclass[12pt]{article}
\renewcommand{\baselinestretch}{1.1}

%%% Add packages here
\usepackage{minted}
\usepackage{listings}
\usepackage{xcolor} % for syntax colors
\usepackage{graphicx} 
%\usepackage{parskip}
\usepackage[table]{xcolor}
\usepackage{amsmath}
%\usepackage{float}
\usepackage{placeins} % prevent float
\usepackage{geometry}
\geometry{legalpaper, portrait, margin=3cm}
\usepackage{makecell}
\usepackage{multirow}
\usepackage{array}
\usepackage{amsfonts}
\usepackage{amsthm}
\usepackage{amssymb}
\usepackage{latexsym}
\usepackage{color}
\usepackage{verbatim}
\usepackage{fancyhdr}
\usepackage{fancybox}
\usepackage{listings}
\usepackage{threeparttable} % for table notes
\usepackage{hologo} % for TeX engine logos
\usepackage{booktabs} % for nice tables
\usepackage{longtable} % for longer tables
\usepackage[table,xcdraw]{xcolor}  % for color in tables
\usepackage{appendix} % for appendix
\usepackage{biblatex}
\addbibresource{references.bib}

\usepackage{amsmath} % for pseudo code
\usepackage{algorithm}
\usepackage{algpseudocode}
\usepackage{subcaption} % For subfigures
\usepackage{threeparttable} % for table notes

\definecolor{codegreen}{rgb}{0,0.6,0}
\definecolor{codegray}{rgb}{0.5,0.5,0.5}
\definecolor{codepurple}{rgb}{0.58,0,0.82}
\definecolor{backcolour}{rgb}{0.95,0.95,0.92}

\lstdefinestyle{mystyle}{
    backgroundcolor=\color{backcolour},   
    commentstyle=\color{codegreen},
    keywordstyle=\color{magenta},
    numberstyle=\tiny\color{codegray},
    stringstyle=\color{codepurple},
    basicstyle=\ttfamily\footnotesize,
    breakatwhitespace=false,         
    breaklines=true,                 
    captionpos=b,                    
    keepspaces=true,                 
    numbers=left,                    
    numbersep=5pt,                  
    showspaces=false,                
    showstringspaces=false,
    showtabs=false,                  
    tabsize=2
}

\lstset{style=mystyle}

%%% Margins
\addtolength{\oddsidemargin}{-.50in}
\addtolength{\evensidemargin}{-.50in}
\addtolength{\textwidth}{1.0in}
\addtolength{\topmargin}{-.40in}
\addtolength{\textheight}{0.80in}

%%% Header
\pagestyle{fancy}
\chead{\groupname} 
\rhead{}
\lhead{}
\cfoot{\thepage}
\renewcommand{\headrulewidth}{0.4pt}

%%%%%%%%%%%%%%%%%%%%%%%%%%%%%%%%%%%%%
\newcommand{\groupname}{Group 4 - Spatial Epidemiology}
\setcounter{secnumdepth}{0}
\begin{document}
\clearpage\thispagestyle{empty}

\begin{titlepage}

\begin{center}
\vspace*{1in}
	% title
	\centering\textbf{\huge{Spatial Epidemiology\\[1cm]
 }} \\[2cm]
	% details
	\Large{\textbf{
	Master of Statistics and Data Science} \\
	Hasselt University\\	
        2024-2025 \\
	}


\vspace*{3cm}
\textbf{\large{Group 4 :}}\\
Winnie Kulei (2469362) \\
Edward Otieno Owinoh (2365191) \\
Loise Kanini (2262458)



\vspace*{2cm}


\vspace*{1.5cm}
\textbf{\large{Lecturers:}}\\
Prof. Dr. Christel Faes \\
Prof. Dr. Thomas Neyens \\

\vspace*{2\baselineskip}
\today

\begin{figure}[b]
   \centering
   \includegraphics[width=8cm]{UHasselt_logog.png}
   \label{fig:Uhasselt}
\end{figure}

\end{center}

\end{titlepage}

\section{Introduction}
In this project, we will investigate the spatial distribution of participation in the screening of colorectal cancer in Flanders. The following data were given: (1) the number of invited individuals per gender, age group (5 age-years per group), and municipality; (2) the number of participating individuals per gender, age group (5 age-years per group), and municipality; and (3) a shape-file of Flanders. The questions of interest are as follows:

\section{1) Are there areas with a lower participation in the screening as compared to participation in the whole of Flanders?}

\subsection{a) Calculate the number of expected participants per
municipality, using the indirect age-standardization with the whole study region (Flanders) as standard population. Calculate this separately for males and females.}

The expected values of indirect standardization are derived by first computing the expected number of participants in age group $j$ per municipality and then summing those numbers to obtain the overall expected number of participants in each municipality, as shown below.
 \begin{align*}
    E_{ij}=r^s_{j}n_{ij} 
 \end{align*}
 where
 $n_{ij}$ is the number of people invited in age group $j$ in a municipality $i$. 
 
 \begin{align*}
    r^s_{j}=\dfrac{\text{number of participants in age group j in Flanders}}{\text{number of people invited in age group j in Flanders}} 
 \end{align*}
  and

\begin{align*}
    E_{i}=\sum_j E_{ij}
 \end{align*}
 The Standardized Participation Ratio (SPR) was then calculated as the ratio of the observed
 number of participants to the expected number of participants. 
 We also computed confidence
 intervals at $0.05$ significance level for the SPRs, assuming the $log(\theta_i)$ is roughly normally distributed, thus ensuring the SPR remains positive.
\begin{align*}
    SPR_{i}=\dfrac{Y_i}{E_i} \quad \text{for} \quad i=1...n
 \end{align*}

\begin{table}[!htbp]
\centering
\caption{A snippet of Standardized Participation Ratios (SPR) by municipality and gender}
\label{tab:SIR_results}
\begin{tabular}{llrrrrrrrr}
\toprule
\textbf{NAAM} & \textbf{Gender} & \textbf{Observed} & \textbf{Expected} & \textbf{Invited\_total} & \textbf{SPR} & \textbf{log\_se} & \textbf{log\_lower} & \textbf{log\_upper} \\
\midrule
Aalst   & female & 2209 & 2435.0 & 4776 & 0.907 & 0.0213 & -0.139  & -0.0558 \\
Aalst   & male   & 1957 & 2260.0 & 4969 & 0.866 & 0.0226 & -0.188  & -0.0996 \\
Aalter  & female &  944 &  832.0 & 1632 & 1.130 & 0.0325 &  0.0621 &  0.1900 \\
Aalter  & male   &  887 &  793.0 & 1741 & 1.120 & 0.0336 &  0.0460 &  0.1780 \\
Aarschot & female & 907 &  892.0 & 1743 & 1.020 & 0.0332 & -0.0485 &  0.0817 \\
Aarschot & male   & 800 &  798.0 & 1745 & 1.000 & 0.0354 & -0.0668 &  0.0718 \\
\bottomrule
\end{tabular}
\end{table}

 
 We observe that the distribution of colorectal cancer cases in Flander has the highest expected
 values in Antwerpen and the lowest expected value found in the Herstappe municipality.
 
\subsection{b) Do some municipalities have a standardized participation ratio (significantly) lower than 1 for males and/or females?}

Based on SPR values, some municipalities have a standardized participation ratio lower than 1 for females and males. Even with the $SPR<1$, we can only tell which municipalities are significantly lower than 1 by looking at the $SPR<1$ whose confidence interval values for their upper limits are also lower than 1. The confidence intervals are calculated based on the error factor method where we assume that assume $log(\theta_i)$ is roughly normally distributed. The table below shows the counts of municipalities with significantly Lower than 1 SPR and higher than 1. 
\begin{table}[H]
\centering
\caption{Summary of municipalities with significantly low or high standardized participation ratios (SPR) by gender}
\label{tab:sir_summary}
\begin{tabular}{lcccc}
\hline
\textbf{Gender} & \textbf{Total Municipalities} & \textbf{Significantly Low } & \textbf{Significantly High }& \textbf{Not Significant} \\
\hline
Female & 300 & \textbf{43} & 92 & 165 \\
Male   & 300 & \textbf{42} & 98 & 160 \\
\hline
\end{tabular}
\end{table}
 These 
$SPR < 1$ indicate less participants are observed in the study population than expected based on the age-specific participation proportions from the standard population. 
%Further tests were conducted to determine whether the true relative risk in these municipalities was lower or not, using the following hypotheses:

%\[
%H_0 : \theta_i = 1 \quad \text{versus} \quad H_a : \theta_i < 1
%\]

%As illustrated in Figures~2, the p-value map shows that some municipalities have a standardized participation ratios (SPRs) significantly lower than 1. 

The above Standardised Participation Ratio (SPR) have separately, for female and male, been visualized in the spatial maps as shown below. 

\begin{figure}[!htbp]
  \centering
  \includegraphics[width=0.8\textwidth]{female participation.png}
  \caption{Spatial distribution of female participants across Flanders.}
  \label{fig:female_map}
\end{figure}

\begin{figure}[!htbp]
  \centering
  \includegraphics[width=0.8\textwidth]{Male participation.png}
  \caption{Spatial distribution of male participants across Flanders.}
  \label{fig:male_map}
\end{figure}


To further determine if the true relative risk in region $i$ is lower or not lower. We test the hypothesis:
\begin{align*}
    H_0 : \theta_i = 1 \quad \text{versus} \quad H_A : \theta < 1
\end{align*}

The resulting $p-values$ that were significant at 5\% level of significance are shown below in the probability maps in Figure~3, separately for male and Females. 
%The the p-value maps shows that some municipalities have a standardized participation ratios (SPRs) significantly lower than 1. 

\begin{figure}[!htbp]
  \centering
  \includegraphics[width=0.8\textwidth]{Probability MAP.png}
  \caption{Male (right) and Female (left) Probability Maps.}
  \label{fig:Probability Map}
\end{figure}


A map of SPR gives a sense of the disease risk although SPRs may be misleading and insufficiently reliable in municipalities with small populations. 


%%%%%%%%%%%%%%%%%%%%%%%%%%%%%%%%%%%%%%%%%%%%%%
\subsection{c) Use the BYM model to model the total number of participating individuals per area (per gender). Estimate the parameter estimates,
relative risks and exceedance probabilities (P $(RR > 1|y))$. What information does the parameter estimates give use? Do any of these
relative risks indicate a lower/higher participation?}
\subsubsection{Adjacency matrix}
An adjacency matrix was defined, so that two regions are neighbours only if they share at least one vertex. For the municipalities without neighbours, the nearest municipality is listed as its neighbour. This is visualized in Figure ~\ref{fig:adj}  where black indicates neighbours, and red indicates a nearest neighbour being chosen.

%adjacency matrix
\begin{figure}[H]
    \centering

    \includegraphics[width=1\linewidth]{Adjc.mat.jpeg}
    \caption{Adjacency matrix}
    \label{fig:adj}
\end{figure}

\subsubsection{Spatial Autocorrelation}
We went further to measure spatial autocorrelation (how strong the tendency is for observations from nearby regions to be more or less alike than observations from regions further apart). This was tested using the Morans I indicator, which compares the value of the variable at any location with the values of all other locations. Under the null hypothesis of no spatial autocorrelation, $I$ is asymptotically normally distributed with mean and variance given by:
\[
E[I] = -\frac{1}{n - 1},
\qquad
\mathrm{Var}[I] = 
\frac{
n^2 \sum_{i} \sum_{j} w_{ij}^2 
+ 3 \left( \sum_{i} \sum_{j} w_{ij} \right)^2
- n \sum_i \left( \sum_j w_{ij} \right)^2
}{
(n^2 - 1) \left( \sum_{i} \sum_{j} w_{ij} \right)^2
},
\]
where $n$ is the number of municipalities and $w_{ij}$ are spatial weights indicating the spatial proximity between regions $i$ and $j$. The results of the Morans I statistic obtained was $I = 0.6004$
%When $I > E[I]$, there is evidence of positive spatial autocorrelation or clustered pattern, while $I < E[I]$ indicates negative spatial autocorrelation. The results obtained were,
%\[
%I = 0.6004, \qquad E[I] = -0.0033,
%\]
suggesting a strong positive spatial autocorrelation in the map.

A Moran’s $I$ scatterplot was also used to visualize the spatial autocorrelation in the data. This plot displays the SPR's of each area against its spatially lagged SPR values (SPRs on average in the neighbours). In figure ~\ref{fig:moran_mc} We observed a positive linear relationship between the SPR's and their spatially lagged values. 
%This means that neighbouring municipalities exhibit similar participation rates for colorectal cancer screening. 
Specifically, we observed that municipalities with high SPR values also have SPRs which is high in the neighbours. Further, a strong local clustering effect is visible in the upper-right quadrant, indicating areas with high local participation are surrounded by other high-participation areas.

\begin{figure}[H]
    \centering
    \includegraphics[width=0.7\textwidth]{moransplot.png}
    \caption{Moran’s I scatterplot showing the SPR values against its spatially lagged SPR values.}
    \label{fig:moran_mc}
\end{figure}

Next, we formally tested the null hypothesis that SPR values are randomly distributed across municipalities following a completely random process using two approaches:  
(1) a normal approximation under independence, which is valid since the number of municipalities is large ($n=300 > 25$); and  
(2) a Monte Carlo permutation test, which provides a more robust inference when normality assumptions may not hold.

Based on the normal approximation test,we observed the p-value obtained is lower than the significance level 0.05 ($p$-value ($< 2.2 \times 10^{-16}$). Then, we reject the null hypothesis and conclude that there is evidence for spatial autocorrelation. The same conclusion was obtained if we use a Monte Carlo approach to assess significance. We observed a p-value of 0.001, less than 0.05, indicating the presence of spatial autocorrelation in the data. This is also shown in Figure~\ref{fig:moran_mc}.

\begin{figure}[H]
    \centering
    \includegraphics[width=0.7\textwidth]{Plotp-pvalue.png}
    \caption{Histogram of Moran’s $I$ values from 999 Monte Carlo simulations under the null hypothesis of spatial independence. The red line indicates the observed Moran’s $I$.}
    \label{fig:moran_mc}
\end{figure}

\subsubsection{BYM model specification}

To account for both correlated and uncorrelated heterogeneity in our data, we fitted the BYM model. This model allows the data to decide how much of the residual is due to spatially structured variation and how much is unstructured overdispersion across the municipalities. 

Let \( O_{i,g} \) denote the observed number of participants in municipality \( i = 1, \ldots, 300 \), and $g$, gender (male or female), assumed to follow a Poisson distribution:
\[
O_{i,g}\ \sim\ \mathrm{Poisson}\big(E_{i,g}\,\theta_{i,g}\big),
\qquad
\log(\theta_{i,g}) \;=\; \alpha_g \;+\; u_i \;+\; v_i,
\]
where \( E_{i,g} \) is the expected counts, and \( \theta_{i,g} \) represents the relative risk (RR) in municipality \( i \) and gender \( g \) compared to the average participation level in Flanders.  \( \alpha_{g} \) denotes the overall intercept, \( v_i \) captures unstructured random effect, and \( u_i \) represents spatially structured random effect. The unstructured component is modeled as exchangeable noise, \( v_i \sim \mathcal{N}(0, \sigma_v^2) \), while the structured component \( u_i \) follows a conditional autoregressive (CAR) prior:
\[
u_i \, | \, u_j, i \neq j, \tau_u^2 \sim \mathcal{N}\left(\bar{u}_i, \frac{1}{\tau_u n_{\delta_i}}\right)
\] where \( n_i \) is the number of neighboring municipalities of area \( i \).
The precision parameters \( \tau_u \) and \( \tau_v \) (inverse of variances \( \sigma_u^2 \) and \( \sigma_v^2 \)) were each assigned Gamma(2, 0.5) priors.  The parameter estimates, relative risks and exceedence probabilities were then estimated.

\subsubsection{Results}

Below is the summary results of our BYM model. From both gender, we notice that $\sigma_u$ is higher than $\sigma_v$ and this implies that our data is more spatially structured as compared to having unstructured heterogeneity and therefore, we say that the variability in the relative ratio is attributed less to uncorrelated heterogeneity than to spatially structured effects. The parameter $\alpha$ is the intercept and gives us the baseline. In our output the intercept of the female model is less than that of male but both are close to $0$. 
\begin{table}[ht!]
\centering
\caption{Results of the BYM model}
\begin{tabular}{llccccc}
\hline \hline
\textbf{Model} & \textbf{Parameter} & \textbf{Mean} & \textbf{SD} & \textbf{Lower CI} & \textbf{Upper CI} & \textbf{WAIC} \\
\hline
Female & $\alpha$     & -0.0293 & 0.0063 & -0.0417 & -0.0169 & 3074.90 \\
 &  sigma.u   &  0.0289 & 0.0039 &  0.0220 &  0.0373 &  \\
 & sigma.v   &  0.0104 & 0.0011 &  0.0084 &  0.0128 &  \\
\hline
Male & $\alpha$     & 0.0528 & 0.0063 & 0.0404 & 0.0652 & 3074.60 \\
& sigma.u   & 0.0287 & 0.0039 & 0.0219 & 0.0370 &  \\
& sigma.v   & 0.0104 & 0.0011 & 0.0084 & 0.0128 & \\
\hline \hline
\end{tabular}
\label{tab:female_model_results}
\end{table}
%writeup for the parameter estimates
%start here...........

%end
In BYM, smoothing occurs through the model's ability to combine both smoothing towards global mean and  towards the neighbours. Here, we notice that both maps exhibit clusters for the various range of relative risk values. Although, whenever we have relative risks for males being $>1.2$ then we experience scattering across the Flander's municipalities. Relative participation in females are more clustered as compared to those of males.  
 

%plots of relative risk and exceedance prob bym model
%RR
\begin{figure}[ht]
\centering
\includegraphics[width=0.48\textwidth]{Female Participation BYM.png}
\includegraphics[width=0.48\textwidth]{Male Participation BYM.png}
\caption{Relative Participation for females (left) and males (right)}
\end{figure}

The exceedance probability seeks to answer the question: What is the chance that an area has higher participation as compared to what we see in the other  areas? It gives the posterior probability that $\theta_i >1$ given data. From our maps below we see that most municipalities show a high participation rates exceeding expectation, especially among males. We also notice that female participation shows more spatial heterogeneity as compared to male, with several municipalities having low exceedance probabilities, implying participation at or below expected levels. This suggests that there is a stronger and more spatially consistent participation among males as compared to females. The lower central region of Flanders for both male and female show that the chance of having more participants than expected is very low while most of the municipalities show that they have relatively high to very high probability of having more participants than expected.


%Exce prob
\begin{figure}[ht]
\centering
\includegraphics[width=0.48\textwidth]{Exceedence Probs Female.png}
\includegraphics[width=0.48\textwidth]{Exceedence Probs Male.png}
\caption{Exceedance probabilities for females (left) and males (right)}
\end{figure}


\section{2) Compare INLA with MCMC: Obtain estimates from the BYM model with both
MCMC and INLA. Compare the estimates as well as the computation time.}

%In order to compare MCMC and INLA in terms of accuracy and computational burden, we fit both models separately for both male and females using the same data. For the MCMC model, a full Bayesian hierarchical model with suitable priors was fitted. Two chains each with  50,000 iterations were selected with the first 5000 iterations discarded as the burnin. Convergence was checked using traceplots and density plots. 

To compare INLA and MCMC in terms of estimation accuracy and computational efficiency, we implemented a BYM model using both approaches, specifying identical non-informative priors for all parameters. Specifically, the precision parameters for the spatially structured ($u_i$) and unstructured ($v_i$) random effects were assigned $\text{Gamma}(2, 0.5)$ priors, while the intercept term $\alpha$ was given a $\text{Normal}(0, \tau_{\alpha}^{-1})$ prior with a hyperprior on the precision parameter $\tau_{\alpha} \sim \text{Gamma}(2, 0.5)$. The models were fitted separately for males and females.

For the MCMC implementation, two chains of 50{,}000 iterations were run, with the first 5{,}000 iterations discarded as burn-in. Convergence was assessed using traceplots and posterior density plots. The INLA implementation was carried out using the \texttt{INLA} package in \texttt{R}, using default Laplace approximations.






%table of comparisons
\begin{table}[H]
\centering
\caption{Comparison of parameter estimates from the BYM model using INLA and MCMC (NIMBLE) by gender}
\begin{tabular}{llrrrrrr}
\toprule \toprule
\textbf{Gender} & \textbf{Parameter} & \multicolumn{3}{c}{\textbf{INLA}} & \multicolumn{3}{c}{\textbf{MCMC}} \\
\cmidrule(lr){3-5} \cmidrule(lr){6-8}
 &  & Mean & low & up & Mean & low & up \\
\midrule
Female & $\alpha$     & -0.0293 & -0.0417 & -0.0169 & -0.0210 & -0.0416 & -0.0165 \\
       & $\sigma^2_u$ &  0.0289 &  0.0220 &  0.0373 &  0.0290 &  0.0220 &  0.0372 \\
       & $\sigma^2_v$ &  0.0104 &  0.0084 &  0.0128 &  0.0105 &  0.0084 &  0.0130 \\
\midrule
Male   & $\alpha$     &  0.0528 &  0.0404 &  0.0652 &  0.0525 &  0.0398 &  0.0650 \\
       & $\sigma^2_u$ &  0.0287 &  0.0219 &  0.0370 &  0.0287 &  0.0219 &  0.0371 \\
       & $\sigma^2_v$ &  0.0104 &  0.0084 &  0.0128 &  0.0104 &  0.0084 &  0.0128 \\
\bottomrule \bottomrule
\end{tabular}
\label{tab:bym_inla_mcmc_gender}
\end{table}

The results as shown in Table {\ref{tab:bym_inla_mcmc_gender}} clearly show that the parameter estimates from MCMC were almost equivalent to INLA across both genders. Specifically, the posterior means, and 95\% credible intervals for the intercept ($\alpha$) and variance parameters ($\sigma^2_u$, $\sigma^2_v$) showed strong agreement. In terms of computational efficiency,the differences were profound. Specifically, for the female dataset, INLA completed estimation in approximately 2.3 seconds, whereas MCMC required about 712 seconds. Similarly, for the male dataset, INLA required 7.6 seconds, while MCMC took about 614 seconds. The overall computational burden of INLA was approximately 80-100 times lower.  
\section{3) Compare the model with other possible models: Poisson-lognormal, Besag, and BYM2. Which model is best fitting to the data? Use two types of diagnostics to compare the model. Give the results of the best fitting model with your conclusions (if different from the BYM model already discussed).}
%Several models were used to describe the spatial distributions of participation of cancer screening. To account for overdispersion, the poisson log-normal model was fitted. In addition, to account for spatial correlation in the data, other models were fitted such as the Besag/Improper Conditional autoregressive (ICAR) model, and the reparameterized form of the BYM model, the BYM2 model. A full bayesian approach was applied in fitting the model in NIMBLE, where identical non informative priors for all parameters were used. Two chains of 50,000 iterations with the first 5000 discarded as the burnin period. Convergence of the MCMC chains was later assessed using traceplots and density plots. The choice of the adjacency matrix is as defined in section \ref{sec:}. To compare the performance between models, the Watanabe Information Criterion (WAIC) and the Mean Absolute Percentage Error (MAPE) between the observed and fitted counts were used. The results of the model comparisons were represented in table \ref{tab:modcomp_mcmc_gender}
%table of the Model comparisons

Several models were used to describe the spatial distribution of participation in cancer screening. To account for overdispersion, a Poisson log-normal model was first fitted. In addition, to accommodate spatial correlation in the data, three spatially explicit models were considered: the improper conditional autoregressive (ICAR/Besag) model, the Besag–York–Mollié (BYM) model, and its reparameterised version, the BYM2 model.

All models were fitted under a fully Bayesian framework in NIMBLE, using identical non- informative priors for all parameters. Two MCMC chains of 50,000 iterations were run for each model, with the first 5,000 iterations discarded as burn-in. Convergence was assessed using trace plots and posterior density plots. The adjacency structure used for the spatial random effects followed the definition provided under the subsection: Adjacency Matrix, where neighbouring areas share common boundaries.

To compare model performance, the Watanabe–Akaike Information Criterion (WAIC), the Mean Absolute Percentage Error (MAPE) and the estimated effective number of parameters (pD) of the models were calculated. 

Table~\ref{tab:modcomp_mcmc_gender} summarises these results for each model and for both genders.
\begin{table}[h!]
\centering
\caption{Model Comparisons using WAIC and MAPE.}
\footnotesize
\begin{tabular}{lcccccc}
%\caption{Model Comparisons using WAIC and MAPE.}
\hline \hline
\textbf{Model} &
\multicolumn{3}{c}{\textbf{Female}} &
\multicolumn{3}{c}{\textbf{Male}} \\
\cline{2-7}
 & WAIC & Pd & MAPE
 & WAIC & Pd & MAPE \\
\hline
\textbf{Poisson--Lognormal (PLN)} &
3115.12 & 157.08 & 36.11 &
3115.90 & 157.42 & 36.13 \\
\textbf{ICAR/BESAG} &
3079.72 & 143.21 & 35.93 &
3080.17 & 143.38 & 35.96 \\
\textbf{BYM} &
3076.22 & 142.63 & 35.86 &
3076.31 & 142.69 & 35.84 \\
\textbf{BYM2} &
3076.27 & 141.97 & 35.90 &
3075.91 & 141.81 & 35.88 \\
\hline \hline
\end{tabular}
\label{tab:modcomp_mcmc_gender}
\end{table}

\noindent
%The results from the table \ref{tab:modcomp_mcmc_gender} indicate the BYM and BYM2 models provide the lowest WAIC and MAPE values indicating better fit for both males and females. The values of the WAIC and MAPE in the ICAR model were slightly higher, but not significantly different. The Poisson log-normal model on the other hand performed worse for both genders with substantially higher WAIC and MAPE values.

%In general, the BYM and BYM2 models yielded the lowest WAIC and MAPE values for both males and females, indicating the best fit of the model. The ICAR model performed only slightly worse, with WAIC values marginally higher than those of the BYM and BYM2 models, while the MAPE values for ICAR, BYM and BYM2 were almost identical. This suggests similar fits to the data. Since the ICAR model is the simplest and most parsimonious formulation, it was preferred as the final model. The Poisson-lognormal model performs worst for both genders, with substantially higher WAIC and MAPE.
In general the results show that the ICAR, BYM, and BYM2 perform substantially better than the Poisson-lognormal model, whose WAIC values are approximately 35 to 40 units higher for both genders. For the ICAR, BYM and BYM2 models, the WAIC and MAPE values were not so different, the BYM and BYM2 models achieving the lowest WAIC for females and males, respectively, though the differences relative to the ICAR model were small with a difference of 3 and 4 in the WAIC values for female and male respectively, and 0.1 difference in the MAPE values. This indicates that all the three models provide similar fits to the data. Since the ICAR model is the simplest and most parsimonious formulation, it was preferred as the final model. The model was specified as follows:

%  ICAR model
Let \( O_{i,g} \) denote the observed number of participants in municipality \( i = 1, \ldots, 300 \), and $g$, gender (male or female), assumed to follow a Poisson distribution:
\[
O_{i,g}\ \sim\ \mathrm{Poisson}\big(E_{i,g}\,\theta_{i,g}\big),
\qquad
\log(\theta_{i,g}) \;=\; \alpha_g \;+\; u_i \,
\]
where \( E_{i,g} \) is the expected counts, and \( \theta_{i,g} \) represents the relative risk (RR) in municipality \( i \) and gender \( g \) compared to the average participation level in Flanders.  \( \alpha_{g} \) denotes the overall intercept, and \( u_i \) represents spatially structured random effect. The structured component \( u_i \) follows a conditional autoregressive (CAR) prior:
\[
u_i \, | \, u_j, i \neq j, \tau_u^2 \sim \mathcal{N}\left(\bar{u}_i, \frac{1}{\tau_u n_{\delta_i}}\right)
\] where \( n_{\delta_i}\) is the number of neighboring municipalities of area \( i \) and \({\delta_i}\) is the set of neighbouring regions of area \( i \).
The precision parameter \( \tau_u \) (inverse of variance \( \sigma_u^2 \) was assigned Gamma(2, 0.5) prior.  The parameter estimates and relative risks were then estimated and presented in table \ref{tab:modcomp_gender}.

%Exponentiating $\alpha$ yields the overall multiplicative participation level relative to the expected counts. The spatial effects $\exp(u_i)$ represent municipality-specific relative risks adjusted for the overall level. Municipalities with $\exp(u_i) > 1$ have higher-than-expected participation, while those with $\exp(u_i) < 1$ have lower-than-expected participation. The posterior mean of $\sigma_u$ quantifies the amount of spatial variation across municipalities.



%that the differences in fit between these three spatial models are very small. Although the BYM and BYM2 models performed marginally better in terms of WAIC, the near-equality of the MAPE values implies that all three models provide very similar fits to the data. Considering this, and the fact that the ICAR model is the simplest and most parsimonious formulation, it may be preferred as the final model. Its conclusions are also consistent with those obtained from the previously discussed BYM model.
%females and males, the models BYM and BYM2 provide the lowest WAIC and among the lowest MAPE values, indicating a better fit. The Poisson-lognormal model performs worst for both genders, with substantially higher WAIC and MAPE. ICAR performs moderately but does not outperform BYM/BYM2.

%table with the parameter estimates
\begin{table}[htbp!]
\centering
\caption{Model comparisons.}
\footnotesize
\begin{tabular}{llcccccccc}
\hline \hline
\textbf{Model} & \textbf{Parameter} &
\multicolumn{4}{c}{\textbf{Female}} &
\multicolumn{4}{c}{\textbf{Male}} \\
\cline{3-10}
 & & Mean & SD & 95\%CI\_low & 95\%CI\_upp
   & Mean & SD & 95\%CI\_low & 95\%CI\_upp \\
\hline
\multicolumn{10}{l}{\textbf{PLN}} \\
 & alpha   & -0.02744 & 0.00957 & -0.04605 & -0.00817
           & 0.05486 & 0.00957 & 0.03656 & 0.07333 \\
 & sigma.v & 0.02683  & 0.00236 & 0.02255 & 0.03180
           & 0.02651 & 0.00233 & 0.02220 & 0.03133 \\
\hline
\multicolumn{10}{l}{\textbf{ICAR}} \\
 & alpha   & -0.02994 & 0.00224 & -0.03436 & -0.02548
           & 0.05213 & 0.00227 & 0.04771 & 0.05675 \\
 & sigma.u & 0.04158  & 0.00380 & 0.03474 & 0.04954
           & 0.04096 & 0.00373 & 0.03421 & 0.04888 \\
\hline
\multicolumn{10}{l}{\textbf{BYM}} \\
 & alpha   & -0.02989 & 0.00603 & -0.04180 & -0.01809
           & 0.05282 & 0.00635 & 0.04017 & 0.06496 \\
 & ratio   & 2.78744  & 0.48018 & 1.95760 & 3.83111
           & 2.76763 & 0.48219 & 1.93596 & 3.78524 \\
 & sigma.u & 0.02889  & 0.00388 & 0.02189 & 0.03720
           & 0.02867 & 0.00395 & 0.02168 & 0.03721 \\
 & sigma.v & 0.01048  & 0.00115 & 0.00844 & 0.01292
           & 0.01048 & 0.00114 & 0.00847 & 0.01285 \\
\hline
\multicolumn{10}{l}{\textbf{BYM2}} \\
 & alpha   & -0.02977 & 0.00281 & -0.03523 & -0.02420
           & 0.05236 & 0.00280 & 0.04683 & 0.05784 \\
 & rho     & 0.95900  & 0.02689 & 0.89657 & 0.99737
           & 0.95847 & 0.02966 & 0.88916 & 0.99786 \\
 & sigma.b & 0.02427  & 0.00249 & 0.01967 & 0.02947
           & 0.02407 & 0.00254 & 0.01942 & 0.02922 \\
\hline \hline
\end{tabular}
\label{tab:modcomp_gender}
\end{table}
The intercepts ($\alpha$) were estimated at $-0.0299$ (female) and $0.0521$ (male). These values were close to zero since we used an indirect standardization, meaning all our risks varies around one. The standard deviation of the random effects were estimated at ($\sigma_u \approx 0.041$) for both males and females,indicating some spatial variation. However, since the values are small, these differences are modest, meaning area 
\(i\) only deviates slightly from its neighbours.

%, means that municipalities  suggesting comparable degrees of spatial variability in participation between municipalities.

\clearpage

\section {4) Is the spatial distribution of participation different between males and females? Fit a model on the data of males and females together that
allows for (1) a different overall participation rate; (2) a different spatial distribution. Write out the model that you fit. Give the results and your conclusion.}

%To examine whether the spatial distribution of participation differs between males and females, we fitted two models within a general spatio-temporal framework, treating gender as a two-level ``pseudo-time'' dimension (female = time~1, male = time~2). Although gender is not a temporal variable, this specification allows us to assess (1) whether the spatial participation maps are shared across genders or (2) whether they differ through a structured space--gender interaction. The models were defined as
%\[
%O_{ig} \sim \text{Poisson}(E_{ig} r_{ig})
%\]
%\[
%\qquad
%\log(r_{ig}) = \alpha_0 + \xi_i + \gamma_g + %\delta_{ig},
%\]
%where index \(i\) denotes the municipality and gender \(g\). The term \(\xi_i\) represents the municipality-specific structured spatial effect, modelled using an intrinsic conditional autoregressive (ICAR) prior, while \(\gamma_g\) is the overall gender effect, modelled as an independent two-level effect. The component \(\delta_{ig}\) captures the space--gender interaction and determines whether the spatial patterns differ between genders.

%In Model~1, the interaction term was removed by setting \(\delta_{ig}=0\). Thus, males and females may differ in overall participation levels through \(\gamma_g\), but they share a common spatial surface \(\xi_i\). In Model~2, the interaction term \(\delta_{ig}\) was modelled as a gender-specific ICAR spatial random effect, with no correlation assumed between the genders. This corresponds to a Type~III Markov random field interaction and yields two distinct spatial maps.

%The  Watanabe-Akaike Information Criterion (WAIC) and the Deviance Information Criteria (DIC) were used to check the best model.

%\subsection{Results}
%\begin{table}[ht!]
%\centering
%\caption{Model comparison results}
%\label{tab:model_comparison}
%\begin{tabular}{lcccc}
%\hline \hline
%\textbf{Model} & \textbf{WAIC} & \textbf{PD (WAIC)} & \textbf{DIC} & \textbf{PD (DIC)} \\
%\hline
%Model 1 & 5406.805 & 146.63 & 5479.69 & 259.08 \\
%Model 2 & 5397.480 & 145.56 & 5477.25 & 265.13 \\
%\hline \hline
%\end{tabular}
%\end{table}

%Model comparison using the Watanabe-Akaike Information Criterion (WAIC) and DIC indicated that Model 2 brovides a better fit than model 1. Although Model 2 has lower WAIC and DIC, the differences are small (less than 10 for WAIC and less than 5 for DIC). Thus, the added complexity of modeling separate spatial surfaces for males and females does not substantially improve fit. Further look on the map of the relative risks for both the males and females, we see the spatial distribution of participation is nearly identical for males and females, confirming there is no difference between the males and females. The estimated intercept parameter $\alpha= 0.0122$corresponds to a relative risk exp()≈1.0123 ≈ +1.23% for females versus males. The 95% credible interval for $\alpha$ is $(−0.0003, 0.0246)$, which exponentiates to approximately (0.9997, 1.0249). Because this interval includes 1, there is no strong evidence of a difference in overall participation between males and females.

%Model 2 attains slightly lower WAIC (5397.48 vs 5406.81) and DIC (5477.25 vs 5479.69), but the differences are small (WAIC difference ≈ 9.3; DIC difference ≈ 2.4) and do not provide strong evidence that the additional complexity is justified. In addition, the posterior interaction variance is very small and Further look on the map of the relative risks for both the males and females, we see the spatial distribution of participation is nearly identical for males and females, confirming there is no difference between the males and females.


%Model comparison using the WAIC and the DIC indicates that Model 2 provides a slightly better fit than Model 1. Specifically, Model 2 achieved a lower WAIC (5397.48 vs. 5406.81) and DIC (5477.25 vs. 5479.69), suggesting improved model adequacy. The effective number of parameters(pD) is comparable between the two models, implying that the improved fit of Model 2 is not due to substantially increased complexity. Overall, these results support the inclusion of gender-specific spatial effects, as captured in Model 2, to more accurately describe spatial variation between males and females.

%%%%%%%%%%%%%%%%%%%%%%%%%%%%%%%%%%%%%%%%%%%%%%%%%%
%Approach 2
To examine whether the spatial distribution of colorectal cancer screening participation differs between males and females, we fitted a Bernardinelli space–time model in which gender was treated as a two-level pseudo-time dimension (female = 1, male = 2). Although gender is not a temporal variable, this specification is appropriate when the number of groups is two and allows the model to distinguish between (1) differences in overall participation levels and (2) differences in the spatial distribution between the groups. The models were defined as
\[
O_{ig} \sim \text{Poisson}(E_{ig} r_{ig})
\]
\begin{equation}
\log(r_{ig}) = \alpha_0 + \xi_i + \beta g \tag{1}
\end{equation}
\begin{equation}
\log(r_{ig}) = \alpha_0 + \xi_i + \beta g + \delta_{ig} \tag{2},
\end{equation}

where index \(i\) denotes the municipality and gender \(g\). The term \(\xi_i\) represents the municipality-specific structured spatial random effects, modelled using an intrinsic conditional autoregressive (ICAR) prior, while \(\beta_g\) is the group linear trend. Since there are only two groups, $\beta$ was interpreted as the difference between the groups. The component \(\delta_{ig}\) is the interaction random effect which determines whether the spatial patterns differ between genders . Model~1 allowed for a gender effect but assumed a shared spatial pattern while, Model~2 extended this by allowing the spatial pattern to differ by gender through a Type~III space–gender interaction type i.e. two independent ICAR spatial structures, one for each gender. The  Watanabe-Akaike Information Criterion (WAIC) and the Deviance Information Criteria (DIC) were used to check the best model. Results are shown in table ~\ref{tab:combined_results} 

\subsection{Results}
Model comparison using WAIC and DIC showed that Model~2 attained slightly lower values (WAIC = 5397.48 vs.\ 5406.81; DIC = 5477.25 vs.\ 5479.69), but the differences were small (WAIC difference ≈ 9.3; DIC difference ≈ 2.4) and did not provide strong evidence in favour of the more complex interaction model. Consistently, the standard deviation of the interaction random effect in Model~2 was extremely small \(\sigma_\delta=0.000095\), perhaps indicating non-significant gender-specific spatial deviations. 
%The estimated interaction variance in Model~2 was extremely small ($\sigma_{\delta} = 0.000095$ with a narrow credible interval), indicating negligible gender-specific spatial variation.
%%%%%%%%%%%%%%%%%%%%%%%%%%%%%%%%%%%%%%%%%%%%%%%%%%%
%table of results
\begin{table}[htbp!]
\centering
\caption{model comparison results}
\label{tab:combined_results}
\begin{tabular}{lcccc}
\hline \hline
\textbf{Model} & \textbf{Mean} & \textbf{St.Dev.} & \textbf{95\% CI Low} & \textbf{95\% CI Upp} \\
\hline
\multicolumn{5}{l}{\textbf{Model 1}} \\
\textt{alpha}        & 0.0128  & 0.0028  & 0.0074  & 0.0183 \\
\textt{beta}    & -0.0014 & 0.0034  & -0.0079 & 0.0052 \\
\textt{sigma.u} & 0.0406  & 0.0037  & 0.0338  & 0.0485 \\
\hline
\multicolumn{5}{l}{\textbf{Model 2}} \\
\textt{alpha}        & 0.0127  & 0.0028  & 0.0073  & 0.0182 \\
\textt{beta}    & -0.0011 & 0.0034  & -0.0078 & 0.0056 \\
\textt{sigma.u} & 0.0405  & 0.0037  & 0.0337  & 0.0483 \\
\textt{sigma.delta} & 0.000095 & 0.000102 & 0.000011 & 0.000367 \\
\hline
\multicolumn{5}{l}{\textbf{Model comparison}} \\
Model 1 & WAIC = 5406.805 & Pd(WAIC) = 146.63 & DIC = 5479.69 & Pd(DIC) = 259.08 \\
Model 2 & WAIC = 5397.480 & Pd(WAIC) = 145.56 & DIC = 5477.25 & Pd(DIC) = 265.13 \\
\hline \hline
\end{tabular}
\end{table}
%%%%%%%%%%%%%%%%%%%%%%%%%%%%%%%%%%%%%%%%%%%%%%%%%%%

The estimated gender effect from Model~1 was $\beta = -0.0014$ (95\% credible interval: $-0.0079$ to $0.0052$). Exponentiating this gives $\exp(\beta) \approx 0.9986$, with a 95\% interval of approximately $(0.9921, 1.0052)$. Similarly, Model~2 produced an estimated effect of $\beta = -0.0011$ (95\% credible interval: $-0.0078$ to $0.0056$), corresponding to $\exp(\beta)\approx 0.9989$ with a 95\% interval of $(0.9922, 1.0056)$. For both models, however, the credible intervals include 1, indicating no evidence of a significant difference in overall participation rates between males and females. 

Visual inspection of the relative risk maps and boxplots shown in Figure~9 further supports this. The spatial relative risk maps are almost identical, showing the same high and low risk municipalities for both genders. This is also visible from the boxplots of the relative risks which overlap for both genders.
%showed that the spatial patterns for males and females were nearly identical.

%Municipalities in the northern regions exhibited higher participation for both genders, and the same geographic structure appeared consistently across the two spatial surfaces produced by Model~2. Thus, even when the model was allowed to estimate gender-specific maps, it did not reveal any systematic differences between them.
%maps and box plot
\begin{figure}[htbp!]
\centering
\includegraphics[width=0.48\textwidth]{rrmaps.png}
\includegraphics[width=0.48\textwidth]{boxplots.png}
\caption{Relative risks across female (1) and male (0) groups estimated by model 2}
\end{figure}
\newpage
%If $Y_{ag}$ is the number of participants amongst $N_{ag}$ individuals that are invited, model\
%$$Y_{ag} \sim Binom(N_{ag},\pi_{ag})$$ with $\pi_{ag}$ the participation rate per area \texttt{a} and gender \texttt{g} group

%To accurately model the spatial trend of the participation rate, we will employ a BYM model. This model includes a normal random effect in the linear predictor (logit of the participation rate) and a spatial random effect to allow for spatial correlation.

%The BYM model structure is defined by the sum of these two components:

%1. The Unstructured Component ($v_i$): This effect accounts for non-spatial heterogeneity. It is modeled using an 'iid' (independent and identically distributed) model, meaning $v_i$ is a zero-mean normally distributed random effect. \\
%2. The Spatially Structured Component ($u_i$): This effect accounts for the genuine spatial correlation (the tendency for nearby regions to have similar rates). It is modeled using a Besag model, which follows a CAR distribution. This spatial structure is based on the defined neighbourhood graph ($g$).

%By combining unstructured and structured components, the model allows local estimates to be a smoothed, weighted average of regional data and observations from nearby regions, effectively separating true geographical influence from purely random local effects.

%The probability $\pi_{ag}$ is modelled via a logit link, as follows:
%$$logit(\pi_{ag})=\beta_0 + \beta_1\text{gender}_a + u_a+ v_a$$





%%%%%%%%%%%%%%%%%%%%%%%%%%%%%%%%%%%%%%%%%%%%%%%%%%


\clearpage
\newpage
\section{Appendix}
\begin{lstlisting}[language=R]
#Homework 1: SPEP
#########QUESTION 1 ####################
# load libraries
library(dplyr)
library(readxl)
library(tidyr) 
library(sf)
library(spdep)
library(tmap)
library(readr) 
library(stringr)
library(purrr)
library(broom)
library(INLA)

# Importing files
data_file <- read_excel("C:/Users/Hp/Downloads/UHasselt 2ND YEAR/Spatial Epidemiology/HW1/DDK - 2022.xlsx")

shp_folder <- st_read("C:/Users/Hp/Downloads/UHasselt 2ND YEAR/Spatial Epidemiology/HW1/shp_folder", quiet = TRUE)
shp_folder<- shp_folder[order(shp_folder$NAAM),]
# reshape 
data_long <- data_file %>%
  pivot_longer(
    cols = -NAAM,
    names_to = c("Type", "Gender", "AgeGroup"),
    names_pattern = "^(Invited|Participant)\\s+(male|female)\\s+(.+)$",
    values_to = "Count"
  ) %>%
  pivot_wider(names_from = Type, values_from = Count) %>%
  rename(Invited = Invited, Participant = Participant) %>%
  mutate(
    Gender = tolower(Gender),
    AgeGroup = str_trim(AgeGroup)
  )

# Flanders-wide age-specific participation rates r_j per gender
rates <- data_long %>%
  group_by(Gender, AgeGroup) %>%
  summarise(
    total_part = sum(Participant, na.rm = TRUE),
    total_inv = sum(Invited, na.rm = TRUE),
    rate = if_else(total_inv > 0, total_part / total_inv, 0),
    .groups = "drop"
  )

# Expected counts per municipality (E_ij = n_ij * r_j) and total E_i
expected_df <- data_long %>%
  left_join(rates %>% select(Gender, AgeGroup, rate), by = c("Gender", "AgeGroup")) %>%
  mutate(E_ij = Invited * rate) %>%
  group_by(NAAM, Gender) %>%
  summarise(
    Observed = sum(Participant, na.rm = TRUE),
    Expected = sum(E_ij, na.rm = TRUE),
    Invited_total = sum(Invited, na.rm = TRUE),
    .groups = "drop"
  )

# SIR and 95% CI
z <- qnorm(0.975)
expected_df <- expected_df %>%
  mutate(
    SIR = if_else(Expected > 0, Observed / Expected, NA_real_),
    log_se = if_else(Observed > 0, sqrt(1 / Observed), NA_real_),
    log_lower = if_else(Observed > 0, log(SIR) - z * log_se, NA_real_),
    log_upper = if_else(Observed > 0, log(SIR) + z * log_se, NA_real_),
    SIR_lower = exp(log_lower),
    SIR_upper = exp(log_upper)
  )

#Municipalities with low/high participation
sig_low <- expected_df %>% filter(SIR_upper < 1)
sig_high <- expected_df %>% filter(SIR_lower > 1)


#Count municipalities with significantly lower or higher participation
summary_counts <- expected_df %>%
  group_by(Gender) %>%
  summarise(
    n_total = n(),
    n_low = sum(SIR_upper < 1, na.rm = TRUE),   # significantly lower
    n_high = sum(SIR_lower > 1, na.rm = TRUE),  # significantly higher
    n_not_sig = n_total - n_low - n_high
  )

summary_counts

#Visualize Female
map_female <- shp_folder %>%
  left_join(
    expected_df %>%
      filter(Gender == "female") %>%
      mutate(Significance = case_when(
        SIR_upper < 1 ~ "Lower than expected",
        SIR_lower > 1 ~ "Higher than expected",
        TRUE ~ "Not significant"
      )),
    by = c("NAAM" = "NAAM")
  )

tmap_mode("plot")
tm_shape(map_female) +
  tm_polygons("Significance", title = "Female participation", palette = "-RdYlGn") +
  tm_layout(legend.outside = TRUE)

#Visualize Male
map_male <- shp_folder %>%
  left_join(
    expected_df %>%
      filter(Gender == "male") %>%
      mutate(Significance = case_when(
        SIR_upper < 1 ~ "Lower than expected",
        SIR_lower > 1 ~ "Higher than expected",
        TRUE ~ "Not significant"
      )),
    by = c("NAAM" = "NAAM")
  )

tmap_mode("plot")
tm_shape(map_male) +
  tm_polygons("Significance", title = "Male participation", palette = "-RdYlGn") +
  tm_layout(legend.outside = TRUE)


#Probability map
# Filter female
female_df <- expected_df %>% filter(Gender == "female")

# Compute probability map
female_df <- female_df %>%
  mutate(p = probmap(Observed, Expected, alternative = "less")$pmap)
male_df <- expected_df %>% filter(Gender == "male")
male_df <- male_df %>%
  mutate(p = probmap(Observed, Expected, alternative = "less")$pmap)
  
# Female map
map_female <- shp_folder %>%
  left_join(female_df, by = c("NAAM" = "NAAM"))

# Male map
map_male <- shp_folder %>%
  left_join(male_df, by = c("NAAM" = "NAAM"))

tmap_mode("plot")

# Female
a1<- tm_shape(map_female) +
  tm_polygons("p", breaks=c(0,0.01,0.05,0.10,0.50,1),palette = 'Red',
              title="P-value (Female)")

# Male
a2<- tm_shape(map_male) +
  tm_polygons("p", breaks=c(0,0.01,0.05,0.10,0.50,1),palette = 'Red',
              title="P-value (Male)")

a1 <- a1 + tm_layout(main.title = "Female Participation Probability Map")

a2 <- a2 + tm_layout(main.title = "Male Participation Probability Map")

tmap_arrange(a1, a2, ncol = 2)


#########QUESTION 3 ####################
library(nimble)
library(coda)

# BYM model
CONVmodel <- nimbleCode({
  for (i in 1:N) {
    O[i] ~ dpois(mu[i])
    log(mu[i]) <- log(E[i]) + alpha + u[i] + v[i]
    RR[i] <- exp(alpha + u[i] + v[i])
    v[i] ~ dnorm(0, tau.v)
  }
  
  # CAR prior for structured effects
  u[1:N] ~ car.normal(adj[1:L], weights[1:L], num[1:N], tau.u, zero_mean=1)
  for (k in 1:L) {
    weights[k] <- 1
  }
  
  # Priors
  alpha ~ dnorm(0, tau.alpha)
  tau.u ~ dgamma(2, 0.5)
  tau.v ~ dgamma(2, 0.5)
  tau.alpha ~ dgamma(2, 0.5)
  
  # Derived quantities
  sigma.u <- 1 / tau.u
  sigma.v <- 1 / tau.v
  ratio <- sigma.u / sigma.v
})

# BYM model with INLA
library(INLA)
nb2INLA("adj", nb.new)
g <- inla.read.graph(filename="adj")

formula <- O ~ f(region.v, model='iid', param=c(2,0.5)) +
  f(region.u, model='besag', graph=g, param=c(2,0.5))

#Model
 model.res <- inla(formula,
 family="Poisson",
 data=my.data,
 E = E,
 control.compute=list(dic=TRUE,waic=TRUE,cpo=TRUE))

##question 4
#### Model 1 

map1$ID<-rep(1:300,each=2)
my.data<-data.frame(Obs=map1$Y, E=map1$E,region.u=map1$ID,region.v=map1$ID,female=map1$female)

formula <- Obs ~ 1 + female +
  f(region.u, model='besag', graph = g, param=c(2,0.5))


model.res <- inla(formula, 
                  family="Poisson", 
                  data=my.data,
                  E = E,
                  control.compute=list(dic=TRUE,waic=TRUE,cpo=TRUE,return.marginals.predictor=TRUE))
                  
model.res$waic$waic # 5412.785 #ICAR = 5406.805
model.res$dic$dic  # 5496.144 # IACR = 5479.685
summary(model.res)

## Model 2
map1$ID<-rep(1:300,each=2)
map1$gender <- ifelse(map1$female == 1, 2, 1)
my.data<-data.frame(Obs=map1$Y,E=map1$E,
region.u=map1$ID,
region.v=map1$ID,
region.u2=map1$ID,
region.v2=map1$ID,
female = map1$female,
gender = map1$gender,
female.region=map1$ID)
                    
formula <- Obs ~ 1 +  female + 
  f(region.u, model='besag', graph = g, param=c(2,0.5)) +
  f(female.region, model="besag", graph =g, group=gender, control.group=list(model="iid"))

model.res <- inla(formula,
                  data = my.data,
                  E=E,family="poisson",
                  control.compute = list(dic = TRUE, waic = TRUE, cpo = TRUE),
                  control.predictor = list(compute = TRUE)
)

summary(model.res)
model.res$waic$waic #WAIC = 5402.76 #ICAR = 5397.455
model.res$dic$dic #DIC = 5493.435   #ICAR = 5477.257


summary(model.res)



\end{lstlisting}





\end{document}
